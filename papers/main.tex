\title{Machine Learning driven DSP-PBE}

\documentclass[12pt]{article}
\usepackage{xspace}

\newcommand{\trainedModel}{\ensuremath{M_{PG}}\xspace}
\newcommand{\processorGroup}{\ensuremath{PG}\xspace}
\newcommand{\codeModel}{\ensuremath{M_n}\xspace}



\begin{document}
\maketitle

\begin{abstract}
In this work we use neural networks to synthesis DSP code from audio examples.
\end{abstract}

\section{Overview}

Our system has three main stages.

First, we use DDSP learning~\cite{} to create a neural network model of the DSP transformation demonstrated by the input-output audio files.

One of the key ideas of DDSP is that every model is built with a directed acyclic graph, called a \textit{ProcessorGroup}.
In this graph. nodes are \textit{Processors} and the edges are the audio signal flow between Processors.
A Processor is a ``core'' DSP element, such as a additive synthesis, noise, or reverb.
A $Processor :: SF\ (\reals^n, *)\ Audio$ is a Signal Function ($SF$, see~\cite{nilsson2002functional} for more on Signal Functions \markk{will need a background section on this}) takes a set of parameters in $\reals^n$ over time, and generates an Audio signal over time.
A Processor additionally consumes (the * in the second part of the tuple) an Audio signal, a collection of Audio signals, or be a generator, and consume no signal.
For a Processor \processor, we use the notation $|\processor|$ to denote the dimension $n$ of the parameters $\reals^n$ for \processor.

The DDSP system works by fixing a ProcessorGroup, $\processorGroup$, then training that over a dataset.
The training process searches for optimal parameters $\reals^n$ for each Processor in the \processorGroup.
The dimensionality of the learning space \markk{is this called the hypothesis space?} is then the sum of the dimensionality of all Processors - $\sum_{i}^{|\processorGroup|} |\processor_i|$.
This yields a trained model $\trainedModel$.


With a trained model, $\trainedModel :: Audio \to Audio$, we want to extract concrete code, $\codeModel :: Audio \to Audio$ that approximates the transformation of the trained model.
In our work, we generate code in the DSP language SuperCollider, though any common DSP language follows the same pattern.
To do this, we generate code that mirrors the underlying structure of $\trainedModel$.

Because the set of possible Processors is small (Additive synthesis, Low Pass Filter, etc), we map each Processor in DDSP to a block of template code in SuperCollider.
The representation of a processor in code, $\codeProcessor :: \reals^n \to SF\ *\ Audio$, has a similar type signature as the Processor in DDSP.
In \codeProcessor however, we do not continuously consume a parameters that change over time.
Instead \codeProcessor must take a single fixed set of parameters that it uses to generate its audio output.

Our task is then the extract a concrete set of parameters for \codeProcessor from the time-varying parameters of \processor.

To connect the modules, we replicate the graph structure of the ProcessingGroup in SuperCollider as the signal flow incode.

We then must extract...



\bibliographystyle{abbrv}
\bibliography{bibliography}

\end{document}
