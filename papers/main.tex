\title{A Very Simple \LaTeXe{} Template}

\documentclass[12pt]{article}
\usepackage{xspace}
\usepackage{xcolor}

\long\def\markk#1{{\color{purple}{\bf Mark: }{\small [#1]}}}

\newcommand{\processorGroup}{\ensuremath{PG}\xspace}
\newcommand{\processor}{\ensuremath{p}\xspace}
\newcommand{\trainedModel}{\ensuremath{M_{\processorGroup}}\xspace}

\newcommand{\codeProcessor}{\ensuremath{c}\xspace}
\newcommand{\codeModel}{\ensuremath{M_c}\xspace}

\newcommand{\reals}{\ensuremath{\mathcal{R}}}



\begin{document}
\maketitle

%\begin{abstract}
In this work we use neural networks to synthesis DSP code from audio examples.
%\end{abstract}

\section{Overview}

Our system has three main stages.

First, we use DDSP learning~\cite{} to create a neural network model of the DSP transformation demonstrated by the input-output audio files.

One of the key contributions of DDSP is that every model is built with a graph structure of \textit{Processors}.
Each Processor in a structure represents a ``core'' DSP element, such as a additive synthesis, noise, or reverb.
The graph represents the signal flow of the audio signal through DSP elements.
This graph is called a \textit{ProcessorGroup}.
The DDSP system works by fixing a ProcessorGroup, $\processorGroup$, then training that over a dataset.
This yields a trained model $\trainedModel$.

With a trained model, $\trainedModel :: Audio \to Audio$, we want to extract concrete code, $\codeModel :: Audio \to Audio$ that approximates the transformation of the trained model.
In our work, we generate code in the DSP language SuperCollider, thought any common DSP language follows the same pattern.
To do this, we look at the underlying structure of $\trainedModel$.

Because the set of Processors is small (less than 10), we map each Processor in DDSP to a block of template code in SuperCollider.
To connect the modules, we replicate the graph structure of the ProcessingGroup in SuperCollider as the signal flow incode.

We then must extract...






\bibliographystyle{abbrv}
\bibliography{main}

\end{document}
